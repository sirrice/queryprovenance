\documentclass{article}
\usepackage{amsmath}
\usepackage{newtxmath}


\begin{document}

Table $R(A,B)$ contains tuple $t_0=(a_0,b_0)$. Query $Q_1$ is applied to the
database: $Q_1 =$ UPDATE R SET $A=u_1$ WHERE $B\leq u_2$. The parameters $u_1$
and $u_2$ are constants specified by the query, but we parameterize them to
allow for their modification based on the complaints.

\smallskip

After the execution of $Q_1$, tuple $t_0$ has become $t_1=(a_1,b_1)$. We will
use attribute-level provenance to represent the resulting values $a_1$ and
$b_1$. For what follows, I use the linearization logic from section 4.4 of the
Tiresias paper.

\smallskip


Let $X$ be a Boolean variable that is true if the WHERE clause of $Q_1$
affects $t_0$: $X=(b_0\leq u_2)$. We will map this to integer variable $x$,
following the logic of the Tiresias provenance representation: $x\otimes
b_0\leq x\otimes u_2$.  We assign new, real variables $v_1= x\otimes
b_0$ and $v_2=x\otimes u_2$.

\smallskip

To linearize the expressions of $v_1$ and $v_2$, we use the following constraints (same as in Tiresias, page 8):

\begin{minipage}{0.7\textwidth}
    \begin{minipage}[t]{0.2\textwidth}
        \begin{align*}
            &v_1\le b_0\\
            &v_1\le xM\\
            &v_1\ge b_0 - (1-x)M
        \end{align*}
    \end{minipage}
    \hspace{4em}
    \begin{minipage}[t]{0.2\textwidth}
        \begin{align*}
            &v_2\le u_2\\
            &v_2\le xM\\
            &v_2\ge u_2 - (1-x)M
        \end{align*}
    \end{minipage}
\end{minipage}

\smallskip

Here, $M$ denotes a very large number (upper bound for the $u_i$).

\smallskip

In $t_1$ the attribute $a_1$ can be expressed: $a_1=x\otimes u_1 +
(1-x)\otimes a_0$, meaning that $a_1$ takes the value $u_1$ if $x=1$, and the
value $a_0$ if $x=0$.  We can now assign: $v_3= x\otimes
u_1$ and $v_4=(1-x)\otimes a_0$, and similarly linearize them:

\begin{minipage}{0.7\textwidth}
    \begin{minipage}[t]{0.2\textwidth}
        \begin{align*}
            &v_3\le u_1\\
            &v_3\le xM\\
            &v_3\ge u_1 - (1-x)M
        \end{align*}
    \end{minipage}
    \hspace{4em}
    \begin{minipage}[t]{0.2\textwidth}
        \begin{align*}
            &v_4\le a_0\\
            &v_4\le (1-x)M\\
            &v_4\ge a_0 - xM
        \end{align*}
    \end{minipage}
\end{minipage}

So, now, we can express the transformation of $t_0$ through query $Q_1$ with a
program with 1 integer variable ($x$) and 8 real variables ($u_1$, $u_2$,
$v_1$, $v_2$, $v_3$, $v_4$, $a_1$, $b_1$). $a_0$ and $b_0$ are constants in
this case, as they are the original values of $t_0$, but they could also be
variables if $t_0$ was derived from another query.

Overall, we have a mixed integer program with the constraints:
\begin{align*}
    &b_1=b_0\\
    &a_1 = v_3+v_4\\
    &0\leq x\leq 1\\
    &v_1\le b_0\\
    &v_1\le xM\\
    &v_1\ge b_0 - (1-x)M\\
    &v_2\le u_2\\
    &v_2\le xM\\
    &v_2\ge u_2 - (1-x)M\\
    &v_3\le u_1\\
    &v_3\le xM\\
    &v_3\ge u_1 - (1-x)M\\
    &v_4\le a_0\\
    &v_4\le (1-x)M\\
    &v_4\ge a_0 - xM
\end{align*}

\smallskip 

This is for the transformation of a single tuple ($t_0$) by a single query.
Similar constraints would be derived for the other tuples, and other queries.
In the end, based on the complaints, we can assign values to the corresponding
variables (e.g., $a_1=5$), and solve the program for the parameters $u_i$,
which will give us a correction. The optimization objective can be set so that
the sum of the differences from the query parameters is minimized.

\smallskip

Since the provenance maintains the transformations, there wouldn't be a need
for rolling back the database instance.


\end{document}