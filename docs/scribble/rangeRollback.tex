\documentclass{article}
\usepackage{amsmath}
\usepackage{newtxmath}


\begin{document}

Consider starting and ending database states $D_s$ and $D_e$ where
$D_e = Q(D_s)$ is the result of executing query $Q$ on the starting
state.  Let each database $D_i$ contain a single tuple $t_i$, where
each attribute $t_i.a_j$ is has value $v_{i,j}$.  In addition, each
attribute $t_i.a_j$ is tagged with a range that defines the values
the attribute can take $r_i.a_j = [min_{i,j}, max_{i,j}]$.

For the end state $D_e$, the values are all well defined, so the value ranges are
defined as $r_i.a_j = [v_{i,j}, v_{i,j}]$.  The question is, when rolling back
across different types of queries, how do we set the ranges?.

\subsection{Update Queries}

Consider the following UPDATE query with 3 SET clauses $s_1, s_2, s_3$ repsectively,
and one WHERE clause $w$:

\[
\text{SET } a=1 \text{ AND } b=f(b) \text{ AND } c=d+1 \text{ WHERE } e \in [1, 10] 
\]

Given the attribute values in $D_e$, we can infer the following
about the ranges for the tuple $t_s$ in the starting database $D_s$:

\begin{enumerate}
\item \[r_{s,a} = \begin{cases}
[-\infty, \infty] & \mbox{if } v_{e,e} \in [1, 10]\\
[1, 1]            & \text{else}
\end{cases}\]
\item $$r_{s,b} = [min(B), max(B)] \mbox{ where } B = \{ f^{-1}(v) | v \in r_{e,b} \}$$.\\
      This is simpler to compute if $f$ is monotonic, in which case $r_{s,b}$ can be computed 
      using only the min, max information of the range $r_{e,b}$.
\item $$r_{s,c} = [min_{e,c}-1, max_{e,c}-1]$$
      This is a special case of $r_{s,b}$, where we can use the range information directly.
\item \[r_{s,e} = \begin{cases}
[-\infty, \infty] - [1,10] & \mbox{if } v_{e,a} \ne 1\\
[1,10]                     & \mbox{if } v_{e,a} = 1 \mbox{ AND } v_{s,a} \ne 1\\
r_{s,e}                    & \mbox{else }
\end{cases}
\]
Notice in this case we make some assumptions about the accuracy of $t_s.a$'s initial value to infer
information about $e$'s correct range.
\end{enumerate}

A more complex query is one that SETs the same attribute that is used in the WHERE clause:

\[
\text{SET } a=0 \text{ WHERE } a \in [1, 10] 
\]


$$
\begin{cases}
  r_{s,a} = [0,10]              & \mbox{if } v_{e,a} =   0\\
  \text{ERROR}                  & \mbox{if } v_{e,a} \in [1, 10]\\
  r_{s,a} = [v_{e,a}, v_{e,a}]  & \mbox{else}
\end{cases}
$$

\end{document}