%!TEX root = ../main.tex


\begin{figure*}[t]
    \begin{minipage}[t]{0.28\textwidth}
         \vspace{0pt} 
         \centering
        \begin{tabular}{llll}
            \multicolumn{4}{l}{\emph{Taxes}: $D_0$}\\
            \toprule
            \textbf{ID}  & \textbf{rate}  & \textbf{income}    & \textbf{owed}\\
            \midrule
            $t_1$   & 10    & \$9500    & \$950\\
            $t_2$   & 25    & \$90000   & \$22500\\
            $t_3$   & 25    & \$86000   & \$21500\\
            \bottomrule
            \\
        \end{tabular}
    \end{minipage}
    \begin{minipage}[t]{0.43\textwidth}
         \vspace{0pt} 
         \centering
        \begin{tabular}{|p{1ex}l|}
            \multicolumn{2}{l}{\emph{Query log}: $\mathcal{Q}$}\\
            % \toprule
            \hline
            $q_1$: & \texttt{\small UPDATE Taxes SET rate = 30}\\
                   & \texttt{\small WHERE income >= \color{red}{85700}} \\
            
            $q_2$: & \texttt{\small UPDATE Taxes SET owed = income*rate/100}\\
            
            $q_3$: & \texttt{\small INSERT INTO Taxes}\\ 
                   & \texttt{\small VALUES (4, 25, 86500, 21625)}\\
            \hline
            % \bottomrule
        \end{tabular}
    \end{minipage}
    \begin{minipage}[t]{0.28\textwidth}
         \vspace{0pt} 
         \centering
        \begin{tabular}{llll}
            \multicolumn{4}{l}{\emph{Taxes}: $D_3$}\\
            \toprule
            \textbf{ID}  & \textbf{rate}  & \textbf{income}    & \textbf{owed}\\
            \midrule
            $t_1$   & 10    & \$9500    & \$950\\
            $t_2$   & 30    & \$90000   & \$27000\\
            \rowcolor{mid-gray}
            $t_3$   & \color{red}{30}    & \$86000   & \color{red}{\$25800}\\
            $t_4$   & 25    & \$86500   & \$21625\\
            \bottomrule
        \end{tabular}
    \end{minipage}

    \caption{A recent change in tax rate brackets calls for a tax rate of 30\% for those with income above \$87500.  The accounting department issues query $q_1$ to implement the new policy, but the predicate of the WHERE clause condition transposed two digits of the income value.  As a result, the tax rate of $t_3$ was increased incorrectly.  Query $q_2$ that calculates the amount owed based on the corresponding tax rate and income propagates the error.  The mistake is further obscured by query $q_3$, which inserts a tuple with similar income and the correct tax rate.}
    \label{fig:example}
\end{figure*}