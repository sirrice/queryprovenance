\section{Basic Solution}

In this section we will introduce a simple approach to solve Prob-Complete.


Given $D^'_n = C(D_n)$, able to roll back to fixed intermediate state $D^'_i$.


\subsection{Single-Query Case}

In this section, we walk through the core techniques in the case of a query sequence containing a single query.
\xxx{Walk through solution for each error type for each query type.}

Inserts and deletes are straight forward, however modifying WHERE clause is harder because
\xxx{why?}.  To deal with this challenge, we tried three algorithmic approaches.

\stitle{Constraint-solver (CPLEX)}
Describe how to encode problem as CPLEX

\stitle{Bounding Box}


\stitle{Decision Tree}
Describe how to encode as decision tree, what algorithm, and how to interpret learned tree.

Also, how to encode constraint that structure needs to be the same.

\subsubsection{Comparing these approaches}

To understand the tradeoffs between these approachse, we ran a simple experiment for a single query.

Here are the results.


CPLEX takes the longest, however produces exact results.  However, it fails to produce any results if there are conflicting complaints.
Decision tree is fastest, however the quality of the results are poor.
Bounding box is similar to CPLEX in quality, however it generates results even if there are conflicting complaints.


\section{Prob-Complete}

Given the above, the algorithm for solving the complete complaint set problem is straightforward (Algorithm~\ref{alg:basic}).
We can simply try each query and return the one for which the best solution is found.

\section{Incomplete Complaints}

Incomplete complaints are a challenge because the exesting algorithms fail.  We ran a simple experiment
where the complaint set contains only M\% of the true complaint set, and has N randomly generated erroneous complaints.
Figure~\ref{} shows that none of the algorithms work.

Describe assumptions for why a cleaning-based approach makes sense.

We use a cleaning-based method to deal with false positives, and some magic to deal with false negatives.

\subsection{False Positives}

Describe density, bi-partite graph, consistency, tuples affected scores.  Which ones work well, which ones don't.

\subsection{False Negatives}



\section{Multi-query Resolution}

We use a dynamic programming-based algorithm to support multiple queries.
